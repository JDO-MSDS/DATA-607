% Options for packages loaded elsewhere
\PassOptionsToPackage{unicode}{hyperref}
\PassOptionsToPackage{hyphens}{url}
\documentclass[
]{article}
\usepackage{xcolor}
\usepackage[margin=1in]{geometry}
\usepackage{amsmath,amssymb}
\setcounter{secnumdepth}{-\maxdimen} % remove section numbering
\usepackage{iftex}
\ifPDFTeX
  \usepackage[T1]{fontenc}
  \usepackage[utf8]{inputenc}
  \usepackage{textcomp} % provide euro and other symbols
\else % if luatex or xetex
  \usepackage{unicode-math} % this also loads fontspec
  \defaultfontfeatures{Scale=MatchLowercase}
  \defaultfontfeatures[\rmfamily]{Ligatures=TeX,Scale=1}
\fi
\usepackage{lmodern}
\ifPDFTeX\else
  % xetex/luatex font selection
\fi
% Use upquote if available, for straight quotes in verbatim environments
\IfFileExists{upquote.sty}{\usepackage{upquote}}{}
\IfFileExists{microtype.sty}{% use microtype if available
  \usepackage[]{microtype}
  \UseMicrotypeSet[protrusion]{basicmath} % disable protrusion for tt fonts
}{}
\makeatletter
\@ifundefined{KOMAClassName}{% if non-KOMA class
  \IfFileExists{parskip.sty}{%
    \usepackage{parskip}
  }{% else
    \setlength{\parindent}{0pt}
    \setlength{\parskip}{6pt plus 2pt minus 1pt}}
}{% if KOMA class
  \KOMAoptions{parskip=half}}
\makeatother
\usepackage{color}
\usepackage{fancyvrb}
\newcommand{\VerbBar}{|}
\newcommand{\VERB}{\Verb[commandchars=\\\{\}]}
\DefineVerbatimEnvironment{Highlighting}{Verbatim}{commandchars=\\\{\}}
% Add ',fontsize=\small' for more characters per line
\usepackage{framed}
\definecolor{shadecolor}{RGB}{248,248,248}
\newenvironment{Shaded}{\begin{snugshade}}{\end{snugshade}}
\newcommand{\AlertTok}[1]{\textcolor[rgb]{0.94,0.16,0.16}{#1}}
\newcommand{\AnnotationTok}[1]{\textcolor[rgb]{0.56,0.35,0.01}{\textbf{\textit{#1}}}}
\newcommand{\AttributeTok}[1]{\textcolor[rgb]{0.13,0.29,0.53}{#1}}
\newcommand{\BaseNTok}[1]{\textcolor[rgb]{0.00,0.00,0.81}{#1}}
\newcommand{\BuiltInTok}[1]{#1}
\newcommand{\CharTok}[1]{\textcolor[rgb]{0.31,0.60,0.02}{#1}}
\newcommand{\CommentTok}[1]{\textcolor[rgb]{0.56,0.35,0.01}{\textit{#1}}}
\newcommand{\CommentVarTok}[1]{\textcolor[rgb]{0.56,0.35,0.01}{\textbf{\textit{#1}}}}
\newcommand{\ConstantTok}[1]{\textcolor[rgb]{0.56,0.35,0.01}{#1}}
\newcommand{\ControlFlowTok}[1]{\textcolor[rgb]{0.13,0.29,0.53}{\textbf{#1}}}
\newcommand{\DataTypeTok}[1]{\textcolor[rgb]{0.13,0.29,0.53}{#1}}
\newcommand{\DecValTok}[1]{\textcolor[rgb]{0.00,0.00,0.81}{#1}}
\newcommand{\DocumentationTok}[1]{\textcolor[rgb]{0.56,0.35,0.01}{\textbf{\textit{#1}}}}
\newcommand{\ErrorTok}[1]{\textcolor[rgb]{0.64,0.00,0.00}{\textbf{#1}}}
\newcommand{\ExtensionTok}[1]{#1}
\newcommand{\FloatTok}[1]{\textcolor[rgb]{0.00,0.00,0.81}{#1}}
\newcommand{\FunctionTok}[1]{\textcolor[rgb]{0.13,0.29,0.53}{\textbf{#1}}}
\newcommand{\ImportTok}[1]{#1}
\newcommand{\InformationTok}[1]{\textcolor[rgb]{0.56,0.35,0.01}{\textbf{\textit{#1}}}}
\newcommand{\KeywordTok}[1]{\textcolor[rgb]{0.13,0.29,0.53}{\textbf{#1}}}
\newcommand{\NormalTok}[1]{#1}
\newcommand{\OperatorTok}[1]{\textcolor[rgb]{0.81,0.36,0.00}{\textbf{#1}}}
\newcommand{\OtherTok}[1]{\textcolor[rgb]{0.56,0.35,0.01}{#1}}
\newcommand{\PreprocessorTok}[1]{\textcolor[rgb]{0.56,0.35,0.01}{\textit{#1}}}
\newcommand{\RegionMarkerTok}[1]{#1}
\newcommand{\SpecialCharTok}[1]{\textcolor[rgb]{0.81,0.36,0.00}{\textbf{#1}}}
\newcommand{\SpecialStringTok}[1]{\textcolor[rgb]{0.31,0.60,0.02}{#1}}
\newcommand{\StringTok}[1]{\textcolor[rgb]{0.31,0.60,0.02}{#1}}
\newcommand{\VariableTok}[1]{\textcolor[rgb]{0.00,0.00,0.00}{#1}}
\newcommand{\VerbatimStringTok}[1]{\textcolor[rgb]{0.31,0.60,0.02}{#1}}
\newcommand{\WarningTok}[1]{\textcolor[rgb]{0.56,0.35,0.01}{\textbf{\textit{#1}}}}
\usepackage{graphicx}
\makeatletter
\newsavebox\pandoc@box
\newcommand*\pandocbounded[1]{% scales image to fit in text height/width
  \sbox\pandoc@box{#1}%
  \Gscale@div\@tempa{\textheight}{\dimexpr\ht\pandoc@box+\dp\pandoc@box\relax}%
  \Gscale@div\@tempb{\linewidth}{\wd\pandoc@box}%
  \ifdim\@tempb\p@<\@tempa\p@\let\@tempa\@tempb\fi% select the smaller of both
  \ifdim\@tempa\p@<\p@\scalebox{\@tempa}{\usebox\pandoc@box}%
  \else\usebox{\pandoc@box}%
  \fi%
}
% Set default figure placement to htbp
\def\fps@figure{htbp}
\makeatother
\setlength{\emergencystretch}{3em} % prevent overfull lines
\providecommand{\tightlist}{%
  \setlength{\itemsep}{0pt}\setlength{\parskip}{0pt}}
\usepackage{bookmark}
\IfFileExists{xurl.sty}{\usepackage{xurl}}{} % add URL line breaks if available
\urlstyle{same}
\hypersetup{
  pdftitle={Data 607 TidyVerse CREATE assignment},
  pdfauthor={Joao De Oliveira @JDO-MSDS},
  hidelinks,
  pdfcreator={LaTeX via pandoc}}

\title{Data 607 TidyVerse CREATE assignment}
\author{Joao De Oliveira @JDO-MSDS}
\date{October 26, 2025}

\begin{document}
\maketitle

\subsection{Overview}\label{overview}

A short, reproducible vignette demonstrating key \textbf{tidyverse}
verbs such as select, rename, mutate, filter, arrange, slice\_head,
group\_by, ggplot, write\_csv, with the
\textbf{fivethirtyeight::college\_recent\_grads} dataset.

\subsection{1. Setup}\label{setup}

\begin{Shaded}
\begin{Highlighting}[]
\FunctionTok{library}\NormalTok{(tidyverse)        }
\FunctionTok{library}\NormalTok{(fivethirtyeight)  }
\end{Highlighting}
\end{Shaded}

\subsection{2. Data and Glimpse}\label{data-and-glimpse}

\begin{Shaded}
\begin{Highlighting}[]
\FunctionTok{data}\NormalTok{(}\StringTok{"college\_recent\_grads"}\NormalTok{, }\AttributeTok{package =} \StringTok{"fivethirtyeight"}\NormalTok{)}
\FunctionTok{glimpse}\NormalTok{(college\_recent\_grads)}
\end{Highlighting}
\end{Shaded}

\begin{verbatim}
## Rows: 173
## Columns: 21
## $ rank                        <int> 1, 2, 3, 4, 5, 6, 7, 8, 9, 10, 11, 12, 13,~
## $ major_code                  <int> 2419, 2416, 2415, 2417, 2405, 2418, 6202, ~
## $ major                       <chr> "Petroleum Engineering", "Mining And Miner~
## $ major_category              <chr> "Engineering", "Engineering", "Engineering~
## $ total                       <int> 2339, 756, 856, 1258, 32260, 2573, 3777, 1~
## $ sample_size                 <int> 36, 7, 3, 16, 289, 17, 51, 10, 1029, 631, ~
## $ men                         <int> 2057, 679, 725, 1123, 21239, 2200, 2110, 8~
## $ women                       <int> 282, 77, 131, 135, 11021, 373, 1667, 960, ~
## $ sharewomen                  <dbl> 0.1205643, 0.1018519, 0.1530374, 0.1073132~
## $ employed                    <int> 1976, 640, 648, 758, 25694, 1857, 2912, 15~
## $ employed_fulltime           <int> 1849, 556, 558, 1069, 23170, 2038, 2924, 1~
## $ employed_parttime           <int> 270, 170, 133, 150, 5180, 264, 296, 553, 1~
## $ employed_fulltime_yearround <int> 1207, 388, 340, 692, 16697, 1449, 2482, 82~
## $ unemployed                  <int> 37, 85, 16, 40, 1672, 400, 308, 33, 4650, ~
## $ unemployment_rate           <dbl> 0.018380527, 0.117241379, 0.024096386, 0.0~
## $ p25th                       <dbl> 95000, 55000, 50000, 43000, 50000, 50000, ~
## $ median                      <dbl> 110000, 75000, 73000, 70000, 65000, 65000,~
## $ p75th                       <dbl> 125000, 90000, 105000, 80000, 75000, 10200~
## $ college_jobs                <int> 1534, 350, 456, 529, 18314, 1142, 1768, 97~
## $ non_college_jobs            <int> 364, 257, 176, 102, 4440, 657, 314, 500, 1~
## $ low_wage_jobs               <int> 193, 50, 0, 0, 972, 244, 259, 220, 3253, 3~
\end{verbatim}

\textbf{Variables of interest}

\begin{itemize}
\tightlist
\item
  major\_category: Category of the major (e.g., Engineering, Business)
\item
  median: Median earnings of full-time, year-round workers
\item
  p25th, p75th: 25th and 75th percentile of earnings
\item
  sharewomen: Proportion of women in the major
\item
  unemployment\_rate: Unemployment rate among recent grads
\end{itemize}

\subsection{3. Cleaning \& Selecting Columns (dplyr::select and
rename)}\label{cleaning-selecting-columns-dplyrselect-and-rename}

\begin{Shaded}
\begin{Highlighting}[]
\NormalTok{grads }\OtherTok{\textless{}{-}}\NormalTok{ college\_recent\_grads }\SpecialCharTok{\%\textgreater{}\%}
  \FunctionTok{select}\NormalTok{(major, major\_category, median, p25th, p75th, sharewomen, unemployment\_rate) }\SpecialCharTok{\%\textgreater{}\%}
  \FunctionTok{rename}\NormalTok{(}\AttributeTok{median\_earn =}\NormalTok{ median, }\AttributeTok{p25 =}\NormalTok{ p25th, }\AttributeTok{p75 =}\NormalTok{ p75th, }\AttributeTok{share\_women =}\NormalTok{ sharewomen)}

\FunctionTok{glimpse}\NormalTok{(grads)}
\end{Highlighting}
\end{Shaded}

\begin{verbatim}
## Rows: 173
## Columns: 7
## $ major             <chr> "Petroleum Engineering", "Mining And Mineral Enginee~
## $ major_category    <chr> "Engineering", "Engineering", "Engineering", "Engine~
## $ median_earn       <dbl> 110000, 75000, 73000, 70000, 65000, 65000, 62000, 62~
## $ p25               <dbl> 95000, 55000, 50000, 43000, 50000, 50000, 53000, 315~
## $ p75               <dbl> 125000, 90000, 105000, 80000, 75000, 102000, 72000, ~
## $ share_women       <dbl> 0.1205643, 0.1018519, 0.1530374, 0.1073132, 0.341630~
## $ unemployment_rate <dbl> 0.018380527, 0.117241379, 0.024096386, 0.050125313, ~
\end{verbatim}

\subsection{4. Mutate, Case When, and
Across}\label{mutate-case-when-and-across}

\begin{Shaded}
\begin{Highlighting}[]
\NormalTok{grads2 }\OtherTok{\textless{}{-}}\NormalTok{ grads }\SpecialCharTok{\%\textgreater{}\%}
  \FunctionTok{mutate}\NormalTok{(}
    \AttributeTok{share\_women\_pct =} \DecValTok{100} \SpecialCharTok{*}\NormalTok{ share\_women,}
    \AttributeTok{earn\_bucket =} \FunctionTok{case\_when}\NormalTok{(}
\NormalTok{      median\_earn }\SpecialCharTok{\textless{}} \DecValTok{30000} \SpecialCharTok{\textasciitilde{}} \StringTok{"\textless{} 30k"}\NormalTok{,}
\NormalTok{      median\_earn }\SpecialCharTok{\textless{}} \DecValTok{40000} \SpecialCharTok{\textasciitilde{}} \StringTok{"30–39k"}\NormalTok{,}
\NormalTok{      median\_earn }\SpecialCharTok{\textless{}} \DecValTok{50000} \SpecialCharTok{\textasciitilde{}} \StringTok{"40–49k"}\NormalTok{,}
      \ConstantTok{TRUE}                \SpecialCharTok{\textasciitilde{}} \StringTok{"50k+"}
\NormalTok{    )}
\NormalTok{  ) }\SpecialCharTok{\%\textgreater{}\%}
  
  \FunctionTok{mutate}\NormalTok{(}\FunctionTok{across}\NormalTok{(}\FunctionTok{c}\NormalTok{(median\_earn, p25, p75), }\SpecialCharTok{\textasciitilde{}} \FunctionTok{round}\NormalTok{(.x, }\SpecialCharTok{{-}}\DecValTok{3}\NormalTok{)))}

\FunctionTok{count}\NormalTok{(grads2, earn\_bucket, }\AttributeTok{sort =} \ConstantTok{TRUE}\NormalTok{)}
\end{Highlighting}
\end{Shaded}

\begin{verbatim}
## # A tibble: 4 x 2
##   earn_bucket     n
##   <chr>       <int>
## 1 30–39k         84
## 2 40–49k         40
## 3 50k+           34
## 4 < 30k          15
\end{verbatim}

\subsection{5. Filter, Arrange, and Slice
Helpers}\label{filter-arrange-and-slice-helpers}

\begin{Shaded}
\begin{Highlighting}[]
\CommentTok{\# Top 10 majors by median earnings}
\NormalTok{top10 }\OtherTok{\textless{}{-}}\NormalTok{ grads2 }\SpecialCharTok{\%\textgreater{}\%}
  \FunctionTok{arrange}\NormalTok{(}\FunctionTok{desc}\NormalTok{(median\_earn)) }\SpecialCharTok{\%\textgreater{}\%}
  \FunctionTok{slice\_head}\NormalTok{(}\AttributeTok{n =} \DecValTok{10}\NormalTok{)}

\NormalTok{top10 }\SpecialCharTok{\%\textgreater{}\%} \FunctionTok{select}\NormalTok{(major, major\_category, median\_earn, unemployment\_rate)}
\end{Highlighting}
\end{Shaded}

\begin{verbatim}
## # A tibble: 10 x 4
##    major                            major_category median_earn unemployment_rate
##    <chr>                            <chr>                <dbl>             <dbl>
##  1 Petroleum Engineering            Engineering         110000            0.0184
##  2 Mining And Mineral Engineering   Engineering          75000            0.117 
##  3 Metallurgical Engineering        Engineering          73000            0.0241
##  4 Naval Architecture And Marine E~ Engineering          70000            0.0501
##  5 Chemical Engineering             Engineering          65000            0.0611
##  6 Nuclear Engineering              Engineering          65000            0.177 
##  7 Actuarial Science                Business             62000            0.0957
##  8 Astronomy And Astrophysics       Physical Scie~       62000            0.0212
##  9 Mechanical Engineering           Engineering          60000            0.0573
## 10 Electrical Engineering           Engineering          60000            0.0592
\end{verbatim}

\subsection{6. Wide ↔ Long
(tidyr::pivot\_longer)}\label{wide-long-tidyrpivot_longer}

\begin{Shaded}
\begin{Highlighting}[]
\NormalTok{earn\_long }\OtherTok{\textless{}{-}}\NormalTok{ grads2 }\SpecialCharTok{\%\textgreater{}\%}
  \FunctionTok{select}\NormalTok{(major, major\_category, median\_earn, p25, p75) }\SpecialCharTok{\%\textgreater{}\%}
  \FunctionTok{pivot\_longer}\NormalTok{(}\AttributeTok{cols =} \FunctionTok{c}\NormalTok{(median\_earn, p25, p75),}
               \AttributeTok{names\_to =} \StringTok{"stat"}\NormalTok{,}
               \AttributeTok{values\_to =} \StringTok{"earn"}\NormalTok{)}

\NormalTok{earn\_long }\SpecialCharTok{\%\textgreater{}\%}
  \FunctionTok{count}\NormalTok{(stat)}
\end{Highlighting}
\end{Shaded}

\begin{verbatim}
## # A tibble: 3 x 2
##   stat            n
##   <chr>       <int>
## 1 median_earn   173
## 2 p25           173
## 3 p75           173
\end{verbatim}

\subsection{7. Grouped Summaries (dplyr::group\_by and
summarise)}\label{grouped-summaries-dplyrgroup_by-and-summarise}

\begin{Shaded}
\begin{Highlighting}[]
\NormalTok{by\_category }\OtherTok{\textless{}{-}}\NormalTok{ grads2 }\SpecialCharTok{\%\textgreater{}\%}
  \FunctionTok{group\_by}\NormalTok{(major\_category) }\SpecialCharTok{\%\textgreater{}\%}
  \FunctionTok{summarise}\NormalTok{(}
    \AttributeTok{majors\_n =} \FunctionTok{n}\NormalTok{(),}
    \AttributeTok{median\_earn\_avg =} \FunctionTok{mean}\NormalTok{(median\_earn, }\AttributeTok{na.rm =} \ConstantTok{TRUE}\NormalTok{),}
    \AttributeTok{share\_women\_avg =} \FunctionTok{mean}\NormalTok{(share\_women\_pct, }\AttributeTok{na.rm =} \ConstantTok{TRUE}\NormalTok{),}
    \AttributeTok{unemp\_avg =} \FunctionTok{mean}\NormalTok{(unemployment\_rate, }\AttributeTok{na.rm =} \ConstantTok{TRUE}\NormalTok{)}
\NormalTok{  ) }\SpecialCharTok{\%\textgreater{}\%}
  \FunctionTok{arrange}\NormalTok{(}\FunctionTok{desc}\NormalTok{(median\_earn\_avg))}

\NormalTok{by\_category }\SpecialCharTok{\%\textgreater{}\%}
  \FunctionTok{slice\_head}\NormalTok{(}\AttributeTok{n =} \DecValTok{8}\NormalTok{)}
\end{Highlighting}
\end{Shaded}

\begin{verbatim}
## # A tibble: 8 x 5
##   major_category              majors_n median_earn_avg share_women_avg unemp_avg
##   <chr>                          <int>           <dbl>           <dbl>     <dbl>
## 1 Engineering                       29          57379.            23.9    0.0633
## 2 Business                          13          43538.            48.3    0.0711
## 3 Computers & Mathematics           11          42727.            31.2    0.0843
## 4 Law & Public Policy                5          42200             48.4    0.0908
## 5 Physical Sciences                 10          41900             50.9    0.0465
## 6 Social Science                     9          37333.            55.4    0.0957
## 7 Agriculture & Natural Reso~       10          36900             40.5    0.0563
## 8 Health                            12          36833.            79.5    0.0659
\end{verbatim}

\subsection{8. Visualization (ggplot2)}\label{visualization-ggplot2}

\begin{Shaded}
\begin{Highlighting}[]
\NormalTok{grads2 }\SpecialCharTok{\%\textgreater{}\%}
  \FunctionTok{mutate}\NormalTok{(}\AttributeTok{major\_category =} \FunctionTok{fct\_lump\_n}\NormalTok{(major\_category, }\AttributeTok{n =} \DecValTok{10}\NormalTok{)) }\SpecialCharTok{\%\textgreater{}\%}
  \FunctionTok{ggplot}\NormalTok{(}\FunctionTok{aes}\NormalTok{(}\AttributeTok{x =} \FunctionTok{fct\_reorder}\NormalTok{(major\_category, median\_earn, }\AttributeTok{.fun =}\NormalTok{ median, }\AttributeTok{na.rm =} \ConstantTok{TRUE}\NormalTok{),}
             \AttributeTok{y =}\NormalTok{ median\_earn)) }\SpecialCharTok{+}
  \FunctionTok{geom\_boxplot}\NormalTok{(}\AttributeTok{outlier.alpha =} \FloatTok{0.3}\NormalTok{) }\SpecialCharTok{+}
  \FunctionTok{geom\_jitter}\NormalTok{(}\AttributeTok{width =} \FloatTok{0.15}\NormalTok{, }\AttributeTok{alpha =} \FloatTok{0.2}\NormalTok{) }\SpecialCharTok{+}
  \FunctionTok{coord\_flip}\NormalTok{() }\SpecialCharTok{+}
  \FunctionTok{labs}\NormalTok{(}\AttributeTok{title =} \StringTok{"Median Earnings by Major Category (Top 10)"}\NormalTok{,}
       \AttributeTok{x =} \StringTok{"Major Category"}\NormalTok{, }\AttributeTok{y =} \StringTok{"Median Earnings (USD)"}\NormalTok{,}
       \AttributeTok{caption =} \StringTok{"Data: fivethirtyeight::college\_recent\_grads"}\NormalTok{)}
\end{Highlighting}
\end{Shaded}

\pandocbounded{\includegraphics[keepaspectratio]{data607_tidyverse_files/figure-latex/plot-earnings-1.pdf}}

\begin{Shaded}
\begin{Highlighting}[]
\CommentTok{\# Relationship between share of women and earnings}
\NormalTok{grads2 }\SpecialCharTok{\%\textgreater{}\%}
  \FunctionTok{ggplot}\NormalTok{(}\FunctionTok{aes}\NormalTok{(}\AttributeTok{x =}\NormalTok{ share\_women\_pct, }\AttributeTok{y =}\NormalTok{ median\_earn, }\AttributeTok{color =}\NormalTok{ major\_category)) }\SpecialCharTok{+}
  \FunctionTok{geom\_point}\NormalTok{(}\AttributeTok{alpha =} \FloatTok{0.6}\NormalTok{, }\AttributeTok{size =} \DecValTok{2}\NormalTok{) }\SpecialCharTok{+}
  \FunctionTok{scale\_x\_continuous}\NormalTok{(}\AttributeTok{labels =}\NormalTok{ scales}\SpecialCharTok{::}\FunctionTok{label\_percent}\NormalTok{(}\AttributeTok{scale =} \DecValTok{1}\NormalTok{)) }\SpecialCharTok{+}
  \FunctionTok{labs}\NormalTok{(}\AttributeTok{title =} \StringTok{"Share of Women vs. Median Earnings"}\NormalTok{,}
       \AttributeTok{x =} \StringTok{"Share of Women (\%)"}\NormalTok{, }\AttributeTok{y =} \StringTok{"Median Earnings (USD)"}\NormalTok{,}
       \AttributeTok{color =} \StringTok{"Category"}\NormalTok{) }
\end{Highlighting}
\end{Shaded}

\pandocbounded{\includegraphics[keepaspectratio]{data607_tidyverse_files/figure-latex/plot-sharewomen-1.pdf}}

\subsection{9. Readr}\label{readr}

\begin{Shaded}
\begin{Highlighting}[]
\CommentTok{\# Save a processed CSV}
\NormalTok{readr}\SpecialCharTok{::}\FunctionTok{write\_csv}\NormalTok{(grads2, }\StringTok{"college\_recent\_grads\_processed.csv"}\NormalTok{)}
\end{Highlighting}
\end{Shaded}

\begin{Shaded}
\begin{Highlighting}[]
\FunctionTok{sessionInfo}\NormalTok{()}
\end{Highlighting}
\end{Shaded}

\begin{verbatim}
## R version 4.5.0 (2025-04-11)
## Platform: aarch64-apple-darwin20
## Running under: macOS Sequoia 15.7.1
## 
## Matrix products: default
## BLAS:   /Library/Frameworks/R.framework/Versions/4.5-arm64/Resources/lib/libRblas.0.dylib 
## LAPACK: /Library/Frameworks/R.framework/Versions/4.5-arm64/Resources/lib/libRlapack.dylib;  LAPACK version 3.12.1
## 
## locale:
## [1] en_US.UTF-8/en_US.UTF-8/en_US.UTF-8/C/en_US.UTF-8/en_US.UTF-8
## 
## time zone: America/New_York
## tzcode source: internal
## 
## attached base packages:
## [1] stats     graphics  grDevices utils     datasets  methods   base     
## 
## other attached packages:
##  [1] fivethirtyeight_0.6.2 lubridate_1.9.4       forcats_1.0.0        
##  [4] stringr_1.5.1         dplyr_1.1.4           purrr_1.0.4          
##  [7] readr_2.1.5           tidyr_1.3.1           tibble_3.2.1         
## [10] ggplot2_3.5.2         tidyverse_2.0.0      
## 
## loaded via a namespace (and not attached):
##  [1] gtable_0.3.6       compiler_4.5.0     tidyselect_1.2.1   scales_1.4.0      
##  [5] yaml_2.3.10        fastmap_1.2.0      R6_2.6.1           labeling_0.4.3    
##  [9] generics_0.1.3     knitr_1.50         pillar_1.10.2      RColorBrewer_1.1-3
## [13] tzdb_0.5.0         rlang_1.1.6        utf8_1.2.4         stringi_1.8.7     
## [17] xfun_0.52          timechange_0.3.0   cli_3.6.5          withr_3.0.2       
## [21] magrittr_2.0.3     digest_0.6.37      grid_4.5.0         rstudioapi_0.17.1 
## [25] hms_1.1.3          lifecycle_1.0.4    vctrs_0.6.5        evaluate_1.0.3    
## [29] glue_1.8.0         farver_2.1.2       rmarkdown_2.29     tools_4.5.0       
## [33] pkgconfig_2.0.3    htmltools_0.5.8.1
\end{verbatim}

\subsubsection{Conclusion}\label{conclusion}

This vignette shows how the Tidyverse simplifies data analysis through
consistent and readable syntax. Using the college\_recent\_grads dataset
from fivethirtyeight, I explored how functions from dplyr, tidyr, and
ggplot2 can transform, summarize, and visualize real-world data
efficiently. Each step---from selecting and mutating variables to
reshaping and plotting---showed how the Tidyverse brings a clear,
logical workflow for data wrangling and exploration. This example
highlights how quickly meaningful insights can be derived when powerful
tools like the Tidyverse are applied to structured data.

\end{document}
